To find the limits of multivariable functions, we can extend the standard delta-epsilon proof to higher dimensions. Interesting thought experiment, but not really relevant.<br><br>
A function is continuous if $\lim_{(x, y) \to (a, b)}f(x, y) = (a, b)$ for all points on the domain. Most functions will be continuous, but just watch for division by 0 and square roots or logs of negatives. If there is a continuity issue, chances are the limit doesn't exist <br><br>
To prove that a limit doesn't exist, use substitution to find two 'paths' that both go through the point of interest but lead to different values. For example, consider $\lim_{(x, y) \to (0, 0)}\frac{y^2\sin^2(x)}{x^4+2y^4}$. If we substitute $y=x$ into the limit (note that this goes through the point $(0, 0)$), we get $\lim_{(x, y=x) \to (0, 0)}\frac{x^2\sin^2(x)}{x^4+2x^4} = \frac{1}{3}$. However, if we evaluate the limit along the path $y=2x$, we get $\lim_{(x, y=2x) \to (0, 0)}\frac{4x^2\sin^2(x)}{x^4+32x^4} = \frac{4}{33}$. Because the limit approaches different values along different paths, we can conclude that it does not exist. Note that we use the axiom that $\sin x = x$ as $ x \to 0$ to evaluate these limits.
