Functions of several variables are functions that take in multiple variables as inputs and return a single output. The easiest example is $f(x, y) = z$, where $f$ takes in some point $P(x, y)$ as input at returns the 'height', $z$, as the output, but these functions can take in as many variables as they want. <br>

These are typically difficult to visualize, and so one way to picture them is to use level curves. For a function $f(x, y) = z$, we fix it to some constant $k$ and then draw all the points $(x, y)$ for which $f(x, y) = k$ in 2D. By doing this for several incremental values of k, we can get a sense for what the function looks like and how steeply it rises. This can also be done for a 4D function $f(x, y, z) = w$ where we draw level surfaces, but this visual isn't as helpful.
