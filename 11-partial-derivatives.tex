The partial derivative of a function is the function's rate of change along the direction of an axis. Technically, it can be defined with the limit definition of a derivative, but practically, you just evaluate its derivative with respect to the partial variable and consider other variables to be a constant.<br><br>
For a function $f(x, y)$, the partial derivative can be written as $f_x$, $f_x(x, y)$, $\frac{\partial f}{\partial x}$, $\frac{\partial z}{\partial x}$<br><br>
Ex. $f(x, y) = x^4y^3 + 8x^2y$<br>
$f_x = 4x^3y^3 + 16xy$ Notice how $y$ is treated as a constant, no different than a numerical coefficient <br>
$f_y = 3x^4y^2 + 8x^2$
