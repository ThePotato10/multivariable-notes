\section{3D Coordinates}

The 3D coordinate system (denoted $\mathbb{R}^3$) extends the 2D coordinate system by adding a $z$-axis that is perpendicular to both the $x$- and $y$-axes. When only two variables are specified in an equation in 3D space, the other variable is free to be whatever it wants. For example, the equation of the line $y = 2x + 3$ in 2D space becomes a plane parallel to the $z$-axis in 3D space. Similarly, all curves in 2D space become surfaces in 3D space and all regions defined by inequalities in 2D space become solid regions in 3D space. For example, $x^2 + y^2 + z^2 \leq 9$ defines a solid sphere of radius 3 in 3D space.

\section{Vectors}

Vectors are a type of data that convey both magnitude and direction. They can be written as either $\overrightarrow{u} = \langle x, y\rangle$ or $\overrightarrow{u} = x\overrightarrow{i} + y\overrightarrow{j}$, where $\overrightarrow{i} = \langle1, 0\rangle, \overrightarrow{j} = \langle0, 1\rangle$. This can easily be generalized to 3D by adding a $z$ component or by adding a $\overrightarrow{k}$ unit vector that equals $\langle0, 0, 1\rangle$. The magnitude of a vector comes from the Pythagorean Theorem, $|\langle x, y, z\rangle| = \sqrt{x^2 + y^2 + z^2}$

\section{Vector Operations}

\subsection{Dot Product}
The dot product, written $\overrightarrow{u} \cdot \overrightarrow{v}$ is a scalar measure of how much two vectors point in the same direction. If they are orthogonal, the dot product equals zero. Given $\overrightarrow{u} = \langlex_1, y_1, z_1\rangle, \overrightarrow{v} = \langlex_2, y_2, z_2\rangle$, the dot product can be computed either as $\langle x_1 \cdot x_2, y_1 \cdot y_2, z_1 \cdot z_2\rangle$, or as $|\overrightarrow{u}||\overrightarrow{v}|cos(\theta)$, where $\theta$ = the angle between the vectors. From this, the angle between two vectors can be computed as $\theta = \cos^{-1}(\frac{\overrightarrow{u} \cdot \overrightarrow{v}}{|\overrightarrow{u}||\overrightarrow{v}|})$
